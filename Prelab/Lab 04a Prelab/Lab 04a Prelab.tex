% AER E 361 Mission Report Template
% Spring 2023
% Template created by Yiqi Liang and Professor Matthew Nelson

% Document Configuration DO NOT CHANGE
\documentclass[12 pt]{article}
% --------------------LaTeX Packages---------------------------------
% The following are packages that are used in this report.
% DO NOT CHANGE ANY OF THE FOLLOWING OR YOUR REPORT WILL NOT COMPILE
% -------------------------------------------------------------------

\usepackage{hyperref}
\usepackage{parskip}
\usepackage{titlesec}
\usepackage{titling}
\usepackage{graphicx}
\usepackage{graphviz}
\usepackage[T1]{fontenc}
\usepackage{titlesec, blindtext, color} %for LessIsMore style
\usepackage{tcolorbox} %for references box
\usepackage[hmargin=1in,vmargin=1in]{geometry} % use 1 inch margins
\usepackage{float}
\usepackage{tikz}
\usepackage{svg} % Allows for SVG Vector graphics
\usepackage{textcomp, gensymb} %for degree symbol
\hypersetup{
	colorlinks=true,
	linkcolor=blue,
	urlcolor=cyan,
}
\usepackage{biblatex}
\addbibresource{lab-report-bib.bib}
\usepackage{amsmath}
\usepackage{listings}
\usepackage{multicol}
\usepackage{array}

\usepackage{hologo} %KYR: for \BibTeX
%\usepackage{algpseudocode}
%\usepackage{algorithm}
% This configures items for code listings in the document
\usepackage{xcolor}

\usepackage{fancyhdr} % Headers/Footers
\usepackage{siunitx} % SI units
\usepackage{csquotes} % Display Quote
\usepackage{microtype} % Better line breaks

\definecolor{commentsColor}{rgb}{0.497495, 0.497587, 0.497464}
\definecolor{keywordsColor}{rgb}{0.000000, 0.000000, 0.635294}
\definecolor{stringColor}{rgb}{0.558215, 0.000000, 0.135316}
\definecolor{mygreen}{rgb}{0,0.6,0}
\definecolor{mygray}{rgb}{0.5,0.5,0.5}
\definecolor{mymauve}{rgb}{0.58,0,0.82}

\lstdefinestyle{customc}{
  belowcaptionskip=1\baselineskip,
  breaklines=true,
  frame=L,
  xleftmargin=\parindent,
  language=C,
  showstringspaces=false,
  basicstyle=\footnotesize\ttfamily,
  keywordstyle=\bfseries\color{green!40!black},
  commentstyle=\itshape\color{purple!40!black},
  identifierstyle=\color{blue},
  stringstyle=\color{orange},
 }

 \lstset{ %
  backgroundcolor=\color{white},   % choose the background color; you must add \usepackage{color} or \usepackage{xcolor}
  basicstyle=\footnotesize,        % the size of the fonts that are used for the code
  breakatwhitespace=false,         % sets if automatic breaks should only happen at whitespace
  breaklines=true,                 % sets automatic line breaking
  captionpos=b,                    % sets the caption-position to bottom
  commentstyle=\color{commentsColor}\textit,    % comment style
  deletekeywords={...},            % if you want to delete keywords from the given language
  escapeinside={\%*}{*)},          % if you want to add LaTeX within your code
  extendedchars=true,              % lets you use non-ASCII characters; for 8-bits encodings only, does not work with UTF-8
  frame=tb,	                   	   % adds a frame around the code
  keepspaces=true,                 % keeps spaces in text, useful for keeping indentation of code (possibly needs columns=flexible)
  keywordstyle=\color{keywordsColor}\bfseries,       % keyword style
  language=Python,                 % the language of the code (can be overrided per snippet)
  otherkeywords={*,...},           % if you want to add more keywords to the set
  numbers=left,                    % where to put the line-numbers; possible values are (none, left, right)
  numbersep=8pt,                   % how far the line-numbers are from the code
  numberstyle=\tiny\color{commentsColor}, % the style that is used for the line-numbers
  rulecolor=\color{black},         % if not set, the frame-color may be changed on line-breaks within not-black text (e.g. comments (green here))
  showspaces=false,                % show spaces everywhere adding particular underscores; it overrides 'showstringspaces'
  showstringspaces=false,          % underline spaces within strings only
  showtabs=false,                  % show tabs within strings adding particular underscores
  stepnumber=1,                    % the step between two line-numbers. If it's 1, each line will be numbered
  stringstyle=\color{stringColor}, % string literal style
  tabsize=2,	                   % sets default tabsize to 2 spaces
  title=\lstname,                  % show the filename of files included with \lstinputlisting; also try caption instead of title
  columns=fixed                    % Using fixed column width (for e.g. nice alignment)
}

\lstdefinestyle{customasm}{
  belowcaptionskip=1\baselineskip,
  frame=L,
  xleftmargin=\parindent,
  language=[x86masm]Assembler,
  basicstyle=\footnotesize\ttfamily,
  commentstyle=\itshape\color{purple!40!black},
}

\lstset{escapechar=@,style=customc}

\titlelabel{\thetitle.\quad}

% From here on out you can start editing your document
\newcommand{\subtitle}[1]{%
  \posttitle{%
    \par\end{center}
    \begin{center}\LARGE#1\end{center}
    \vskip0.5em}%
}

\newcommand{\etal}{\textit{et al}., }
\newcommand{\ie}{\textit{i}.\textit{e}., }
\newcommand{\eg}{\textit{e}.\textit{g}., }

% Define the headers and footers
\setlength{\headheight}{70.63135pt}
\geometry{head=70.63135pt, includehead=true, includefoot=true}
\pagestyle{fancy}
\fancyhead{}\fancyfoot{} % clears the headers/footers
\fancyhead[L]{\textbf{AER E 322}}
\fancyhead[C]{\textbf{Aerospace Structures Pre-Laboratory}\\
			  \textbf{Lab 4a Strain Gage Application and Testing}\\
			  Section 4 Group 2\\
			  Matthew Mehrtens\\
			  \today}
\fancyhead[R]{\textbf{Spring 2023}}
\fancyfoot[C]{\thepage}

\begin{document}
\section*{Question 1} \label{question_1}
\textit{(15 points) Provide your own account about strain gage: what it is, how does it work and how will it be used in this lab.}

Strain gauges, in general, serve to mechanically measure the strain experienced in a material, \ie the normalized change in length of a substance when a load is applied. The strain gauges we are working with in this lab use electrical resistivity as a correlate to the strain in a material. The specific model of strain gauge we are using is known as a ``bonded resistance strain gauge'' which consists of a conductive grid bonded to a backing. In this device, the measured resistivity is linearly correlated with the change in strain; so, the strain on the surface of a material can be found by calculating the change in resistance within an electrical circuit.

In this lab, we will use the strain gauges to measure strain in two different scenarios: a tensile test and a bending test. In the tensile test, front and back strain gauges will be averaged to find the tensile strain. In the bending test, the difference between the front and back gauges will be calculated to find the stress due to bending.

\section*{Question 2} \label{question_2}
\textit{What are the gage factor (3 pts) and resistance (3 pts) of the gage used in this lab? Specify your source.}

From the ``Strain-gage-technical-data.pdf'' provided in the Canvas modules, we can see that the model we will be using, the SGD-5/350-LY13, has a gage factor of \num{2.0}$\pm$\qty{5}{\percent}. From the ``Strain-gage-datasheet-SGD-5-350-LY13.pdf'' provided in the Canvas module, we note that the SGD-5/350-LY13 has a nominal resistance of \qty{350}{\ohm}.

\section*{Question 3} \label{question_3}
\textit{(10 pts) Describe the general procedure of gage installation.}

First, the specimen must be cleaned which involves removing the plastic cover, roughing up the area in the center of the specimen, and using a couple of drops of acetone to clean the center of the sample with a Q-tip. Next, the center is marked on each edge of the sample to help with alignment of the gauge.

To apply the strain gauge, the gauge is stuck to a piece of tape which will be used to assist in alignment of the gauge. Once the gauge is aligned and stuck to the surface of the sample, the tape is peeled back and glue is applied to the surface. The gauge is then pressed back down for \qtyrange{90}{120}{\second} until fully cured. The alignment tape is removed. A piece of tape is applied adjacent to the bottom edge of the gauge underneath the leads to insulate them. Finally, a piece of long tape with a smaller piece stuck to it---creating a small non-sticky area for the gauge---is placed on top of the gauge to fully insulate it.

The gauge resistance is sanity checked, and then the \qty{18}{in} wires are stripped and looped around the leads. The leads and the wires are soldered together and trimmed, and then tape is used to secure the leads and insulate everything. Lastly, the resistance of the gauge is checked one last time to ensure the resistance is reading its nominal value.

\section*{Question 4} \label{question_4}
\textit{(4 pts) According to the lab manual, how much glue will be applied in gage installation?}

According to the manual, only one \textit{small} drop of glue is to be used to bond the strain gauge to the surface of the sample.

\end{document}
