% AER E 361 Mission Report Template
% Spring 2023
% Template created by Yiqi Liang and Professor Matthew Nelson

% Document Configuration DO NOT CHANGE
\documentclass[12 pt]{article}
% --------------------LaTeX Packages---------------------------------
% The following are packages that are used in this report.
% DO NOT CHANGE ANY OF THE FOLLOWING OR YOUR REPORT WILL NOT COMPILE
% -------------------------------------------------------------------

\usepackage{hyperref}
\usepackage{parskip}
\usepackage{titlesec}
\usepackage{titling}
\usepackage{graphicx}
\usepackage{graphviz}
\usepackage[T1]{fontenc}
\usepackage{titlesec, blindtext, color} %for LessIsMore style
\usepackage{tcolorbox} %for references box
\usepackage[hmargin=1in,vmargin=1in]{geometry} % use 1 inch margins
\usepackage{float}
\usepackage{tikz}
\usepackage{svg} % Allows for SVG Vector graphics
\usepackage{textcomp, gensymb} %for degree symbol
\hypersetup{
	colorlinks=true,
	linkcolor=blue,
	urlcolor=cyan,
}
\usepackage{biblatex}
\addbibresource{lab-report-bib.bib}
\usepackage{amsmath}
\usepackage{listings}
\usepackage{multicol}
\usepackage{array}

\usepackage{hologo} %KYR: for \BibTeX
%\usepackage{algpseudocode}
%\usepackage{algorithm}
% This configures items for code listings in the document
\usepackage{xcolor}

\usepackage{fancyhdr} % Headers/Footers
\usepackage{siunitx} % SI units
\usepackage{csquotes} % Display Quote
\usepackage{microtype} % Better line breaks

\definecolor{commentsColor}{rgb}{0.497495, 0.497587, 0.497464}
\definecolor{keywordsColor}{rgb}{0.000000, 0.000000, 0.635294}
\definecolor{stringColor}{rgb}{0.558215, 0.000000, 0.135316}
\definecolor{mygreen}{rgb}{0,0.6,0}
\definecolor{mygray}{rgb}{0.5,0.5,0.5}
\definecolor{mymauve}{rgb}{0.58,0,0.82}

\lstdefinestyle{customc}{
  belowcaptionskip=1\baselineskip,
  breaklines=true,
  frame=L,
  xleftmargin=\parindent,
  language=C,
  showstringspaces=false,
  basicstyle=\footnotesize\ttfamily,
  keywordstyle=\bfseries\color{green!40!black},
  commentstyle=\itshape\color{purple!40!black},
  identifierstyle=\color{blue},
  stringstyle=\color{orange},
 }

 \lstset{ %
  backgroundcolor=\color{white},   % choose the background color; you must add \usepackage{color} or \usepackage{xcolor}
  basicstyle=\footnotesize,        % the size of the fonts that are used for the code
  breakatwhitespace=false,         % sets if automatic breaks should only happen at whitespace
  breaklines=true,                 % sets automatic line breaking
  captionpos=b,                    % sets the caption-position to bottom
  commentstyle=\color{commentsColor}\textit,    % comment style
  deletekeywords={...},            % if you want to delete keywords from the given language
  escapeinside={\%*}{*)},          % if you want to add LaTeX within your code
  extendedchars=true,              % lets you use non-ASCII characters; for 8-bits encodings only, does not work with UTF-8
  frame=tb,	                   	   % adds a frame around the code
  keepspaces=true,                 % keeps spaces in text, useful for keeping indentation of code (possibly needs columns=flexible)
  keywordstyle=\color{keywordsColor}\bfseries,       % keyword style
  language=Python,                 % the language of the code (can be overrided per snippet)
  otherkeywords={*,...},           % if you want to add more keywords to the set
  numbers=left,                    % where to put the line-numbers; possible values are (none, left, right)
  numbersep=8pt,                   % how far the line-numbers are from the code
  numberstyle=\tiny\color{commentsColor}, % the style that is used for the line-numbers
  rulecolor=\color{black},         % if not set, the frame-color may be changed on line-breaks within not-black text (e.g. comments (green here))
  showspaces=false,                % show spaces everywhere adding particular underscores; it overrides 'showstringspaces'
  showstringspaces=false,          % underline spaces within strings only
  showtabs=false,                  % show tabs within strings adding particular underscores
  stepnumber=1,                    % the step between two line-numbers. If it's 1, each line will be numbered
  stringstyle=\color{stringColor}, % string literal style
  tabsize=2,	                   % sets default tabsize to 2 spaces
  title=\lstname,                  % show the filename of files included with \lstinputlisting; also try caption instead of title
  columns=fixed                    % Using fixed column width (for e.g. nice alignment)
}

\lstdefinestyle{customasm}{
  belowcaptionskip=1\baselineskip,
  frame=L,
  xleftmargin=\parindent,
  language=[x86masm]Assembler,
  basicstyle=\footnotesize\ttfamily,
  commentstyle=\itshape\color{purple!40!black},
}

\lstset{escapechar=@,style=customc}

\titlelabel{\thetitle.\quad}

% From here on out you can start editing your document
\newcommand{\subtitle}[1]{%
  \posttitle{%
    \par\end{center}
    \begin{center}\LARGE#1\end{center}
    \vskip0.5em}%
}

\newcommand{\etal}{\textit{et al}., }
\newcommand{\ie}{\textit{i}.\textit{e}., }
\newcommand{\eg}{\textit{e}.\textit{g}., }

% Define the headers and footers
\setlength{\headheight}{70.63135pt}
\geometry{head=70.63135pt, includehead=true, includefoot=true}
\pagestyle{fancy}
\fancyhead{}\fancyfoot{} % clears the headers/footers
\fancyhead[L]{\textbf{AER E 322}}
\fancyhead[C]{\textbf{Aerospace Structures Pre-Laboratory}\\
			  \textbf{Lab 4 Strain Gage Application and Testing}\\
			  Section 4 Group 2\\
			  Matthew Mehrtens\\
			  \today}
\fancyhead[R]{\textbf{Spring 2023}}
\fancyfoot[C]{\thepage}

\begin{document}
\section*{Question 1} \label{question_1}
\textit{(10 points) You were introduced to the concept of calibration. Provide your own account what it is in general.}

When performing the tensile strain test with the Instron machine, the results contain an element of systematic error. This error arises from Newton's third law of motion. While the grips of the Instron are applying a tensile force to the aluminum sample, a force is exerted back on the grips of the Instron and the effects of this force are non-negligible. Calibration is the process of measuring this erroneous reaction force and subsequent elongation, so it can later be removed from the final change in length of the material.

\section*{Question 2} \label{question_2}
\textit{Continuing from Question 1, we will perform a calibration procedure in this lab.  Again, in your own words, describe how it is done (5 pts), explain how does it work (5 pts) and how will it be used in this lab (5 pts).}

To calibrate our strain gauge tensile test, we need to calculate the elongation due to the reaction forces on the Instron machine. To calculate this erroneous reaction force, we run the Instron with a very stiff material and calculate the measured elongation. This measured elongation is assumed to be approximately the elongation due to the reaction force since the stiff material's elongation is negligible. Later, we can calculate the true change in length of the sample, ${\Delta}L_s$, by subtracting the elongation due to the reaction force, $L_g$, from the change in length of the Instron frame, ${\Delta}L_f$. The resulting equation is ${\Delta}L_s={\Delta}L_f-L_g$.

\section*{Question 3} \label{question_3}
\textit{We will measure two sets of strain data for tensile test in this lab.  What are the sources (2 pts)? Describe the experimental procedures to obtain either data set (8 pts).}

One source of strain data will come from the Bluehill software which is tracking the elongation recorded by the Instron machine. The other source of strain data will come directly from the strain gauges attached to the sample, recorded with the LabView software.

To obtain the strain data for the tensile test from the Instron/Bluehill apparatus, load the sample into the Instron and tighten the grips down so the sample is perfectly vertical with no slack. Connect the strain gauge wires on the sample to the Vishay P-3500 Strain Indicator in the Wheatstone half-bridge II configuration. This bridge configuration is created by using a full-bridge configuration with two external resistors.

Before recording, ensure the strain gauges and the Instron are zeroed/balanced. Start recording in the LabView software and then start the test in the BlueHill software. The BlueHill/Instron test will end automatically, after which the LabView recording will need to be manually ended. The LabView software automatically outputs a file containing an array with the strain values. The BlueHill data will need to be located and is in a \texttt{.csv} file format with the time, stress, and elongation values recorded.

To get the strain data from the BlueHill data, additional calculations will be required, including subtracting out the $L_g$ calibration factor from ${\Delta}L_f$ to get ${\Delta}L_s$. Once you have derived ${\Delta}L_s$, you can divide this value by the original length, $L_s$, to get the strain values over time or the stress versus strain curves.

\section*{Question 4} \label{question_4}
\textit{Continuing from Question 3, which one of the two sets of strain data should be more accurate (2 pts)?  Based on what you have studied about this lab, explain why (8pts).}

The strain data from the strain gauges will be more accurate. These gauges are measuring the strain more directly than the Instron machine and have a mucher higher precision than the Instron. In the case of the Instron, we are calculating the reaction error by assuming that when calibrating with a stiff block, the elongation of the block is approximately \qty{0}{m}. In reality, the stiff block does elongate slightly; so, this assumption introduces error into our strain calculations.

\section*{Question 5} \label{question_5}
\textit{(5 pts) Define ``gage length'' in the tensile test.}

Gauge length is defined as the distance between the grips of the Instron testing machine, $L_s$. In a theoretical scenario where this is no reaction force on the grips of the Instron, the change in gauge length, ${\Delta}L_s=L_s'-L_s={\Delta}L_f=L_f'-L_f$. In reality, the stretch of (mainly) the grip due to the reaction force, $L_g$, must be accounted for, hence, the change in the gauge length is really, ${\Delta}L_s={\Delta}L_f-L_g$.

\end{document}
